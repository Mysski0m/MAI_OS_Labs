\section{Выводы}
В ходе выполнения лабораторной работы были приобретены практические навыки создания и использования динамических библиотек.
Были реализованы и отлажены две стратегии взаимодействия с динамическими библиотеками:
\begin{itemize}
    \item \texttt{implicit linking} --— связывание на этапе компиляции (программы \texttt{main\_link} и \texttt{main\_link\_alt}),
    \item \texttt{explicit linking} —-- загрузка библиотек во время выполнения через \texttt{POSIX API} (\texttt{main\_runtime}).
\end{itemize}
Для обеспечения совместимости с \texttt{dlsym} все экспортируемые функции объявлены с использованием \texttt{extern "C"}, что предотвращает \texttt{name mangling} и гарантирует корректное разрешение символов.
Архитектура программы построена по принципу инкапсуляции: системно-зависимый код выделен в отдельный модуль (\texttt{platform/dynlib.cpp}), что повышает читаемость, безопасность (благодаря \texttt{RAII}) и облегчает возможную адаптацию под другие ОС в будущем.
Результаты работы полностью соответствуют требованиям задания и демонстрируют глубокое понимание механизмов динамической загрузки кода, управления памятью и модульного проектирования программ на \texttt{C++}.
\pagebreak