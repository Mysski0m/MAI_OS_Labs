\section{Результаты}

Программа успешно создаёт четыре динамические библиотеки (\texttt{libgcf\_euclid.so},\\ \texttt{libgcf\_naive.so}, \texttt{libtranslation\_binary.so}, \texttt{libtranslation\_ternary.so}), реализующие две функции с двумя различными алгоритмами каждая.\\
Тестовая программа \texttt{main\_link} корректно использует библиотеки, подключённые на этапе линковки, и выдаёт результаты вычислений НОД и перевода чисел в заданную систему счисления.\\
Программа \texttt{main\_runtime} загружает библиотеки во время выполнения с помощью системного \texttt{API} (\texttt{dlopen, dlsym, dlclose}), поддерживает переключение реализаций по команде \texttt{0} и корректно вызывает функции через полученные указатели.\\
Все команды обрабатываются в соответствии с заданным форматом ввода:
\begin{itemize}
    \item \texttt{1 A B} — вычисление НОД,
    \item \texttt{2 X} — перевод числа,
    \item \texttt{0} — переключение реализаций (только в \texttt{main\_runtime}).
\end{itemize}
Память, выделенная в библиотеках, освобождается в вызывающем коде, что предотвращает утечки.
Программы корректно обрабатывают ошибки: отсутствие \texttt{.so}-файлов, отсутствие символов в библиотеках, неверный ввод пользователя. В случае ошибки выводится понятное сообщение, и программа завершается безопасно.
