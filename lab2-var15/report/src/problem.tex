\section{Условие}

{\bfseries Цель работы:} \\
Приобретение практических навыков в:
\begin{itemize}
\item Управление потоками в ОС
\item Обеспечение синхронизации между потоками
\end{itemize}

{\bfseries Задание:} \\
Составить программу на языке Си, обрабатывающую данные в многопоточном режиме. При обработки использовать стандартные средства создания потоков операционной системы (Windows/Unix). Ограничение максимального количества потоков, работающих в один момент времени, должно быть задано ключом запуска вашей программы.
Так же необходимо уметь продемонстрировать количество потоков, используемое вашей программой с помощью стандартных средств операционной системы.
В отчете привести исследование зависимости ускорения и эффективности алгоритма от входных данных и количества потоков. Получившиеся результаты необходимо объяснить.

{\bfseries Вариант:} 15 \\
15.	Есть колода из 52 карт, рассчитать экспериментально (метод Монте-Карло) вероятность того, что сверху лежат две одинаковых карты. Количество раундов задаётся ключом программы.
\pagebreak