\section{Метод решения}

Алгоритм решения задачи:

\begin{enumerate}
\item Пользователь в консоль вводит два параметра командной строки:
    \begin{itemize}
        \item rounds N — общее количество экспериментов (раундов),
        \item max-threads M — максимальное число потоков, выполняемых одновременно.
    \end{itemize}
    Проверяется корректность входных данных (положительные целые числа).

\item Общее число раундов N равномерно распределяется между M потоками.
Остаток от деления N mod M добавляется к первому потоку для обеспечения полноты покрытия.

\item Для каждого потока создаётся объект \texttt{std::thread}.
Каждому потоку передаётся:
    \begin{itemize}
        \item Указатель на функцию обработки (\texttt{card\_function}),
        \item Количество раундов для данного потока (типа \texttt{long long})
    \end{itemize}
\item Для каждого раунда:
    \begin{itemize}
        \item Колода из 52 карт (13 рангов × 4 масти) создаётся заново.
        \item Колода случайным образом перемешивается с использованием \texttt{std::shuffle} и генератора \texttt{std::mt19937}, инициализированного энтропией\\ из \texttt{std::random\_device}.
        \item Проверяется, совпадают ли достоинства первых двух карт.
        \item При совпадении увеличивается локальный счётчик успешных исходов.
    \end{itemize}
\item После завершения всех раундов поток:
    \begin{itemize}
        \item Блокирует глобальный мьютекс (\texttt{std::mutex}) с помощью \texttt{std::lock\_guard},
        \item Обновляет общие счётчики (\texttt{total\_rounds, total\_successes}),
        \item Мьютекс автоматически разблокируется при выходе из области видимости.
    \end{itemize}
\item Главный поток ожидает завершения всех рабочих потоков с помощью метода \texttt{.join()}.
Вычисляется экспериментальная вероятность:

\begin{center}
$P = \frac{total\_rounds}{total\_successes}$
\end{center}
\item После успешного завершения работы программа выводит:
\begin{itemize}
    \item Общее число раундов,
    \item Число успешных исходов,
    \item Экспериментальную вероятность,
    \item Время выполнения в наносекундах и секундах (с использованием \texttt{std::chrono}.
\end{itemize}
\end{enumerate}
\pagebreak
Архитектура программы:

\dirtree{%
.1 lab2-var15/. 
.2 bin/. 
.2 build/.
.2 include/.
.3 deck.h.
.3 monte\_carlo\_card.h.
.2 src/.
.3 deck.cpp.
.3 monte\_carlo\_card.cpp.
.2 main.cpp.
.2 CMakeLists.txt.
}

Ссылки:

\begin{itemize}
\item \url{https://en.cppreference.com/w/cpp/chrono.html}
\item \url{https://en.cppreference.com/w/cpp/thread/thread.html}
\item \url{https://en.cppreference.com/w/cpp/thread/mutex.html}
\item \url{https://en.cppreference.com/w/cpp/thread/lock_guard.html}
\item \url{https://en.cppreference.com/w/cpp/numeric/random/rand.html}
\item \url{https://en.cppreference.com/w/cpp/algorithm/random_shuffle.html}
\item \url{https://en.cppreference.com/w/cpp/numeric/random/mersenne_twister_engine.html}
\item \url{https://en.cppreference.com/w/cpp/algorithm.html}
\item \url{https://en.cppreference.com/w/cpp/header/cstdlib.html}
\end{itemize}

\section{Описание программы}

\texttt{main.cpp} --- точка входа в программу. Выполняет парсинг аргументов командной строки (\texttt{--rounds N}, \texttt{--max-threads M}), проверку их корректности, распределение вычислительной нагрузки между потоками и запуск многопоточного моделирования. После завершения всех потоков выводит статистику: общее число раундов, количество успешных исходов, экспериментальную вероятность и время выполнения.

\texttt{include/deck.cpp} --- модуль работы с колодой карт. Содержит объявление структуры \texttt{Card} и функцию \texttt{CreateDeck()}, создающую стандартную колоду из 52 карт (13 достоинств × 4 масти).

Основные функции:
\begin{itemize} 
    \item \texttt{std::vector<Card> CreateDeck();} --- создаёт стандартную колоду из 52 карт. Колода представлена вектором объектов \texttt{Card}, где каждый объект содержит поле \texttt{denomination} (достоинство карты от 0 до 12). Колода упорядочена по достоинствам (по 4 карты каждого ранга).
\end{itemize}

\texttt{include/monte\_carlo\_card.cpp} --- модуль логики моделирования методом Монте-Карло. Содержит глобальное состояние программы и функцию \texttt{card\_function()}, реализующую вычисления в отдельном потоке.

Основные функции:
\begin{itemize} 
    \item \texttt{void card\_function(long long rounds);} --- выполняет заданное количество раундов в отдельном потоке:
    \begin{itemize}
        \item Для каждого раунда создаёт копию колоды и перемешивает её с использованием \texttt{std::shuffle} и генератора \texttt{std::mt19937}.
        \item Проверяет, совпадают ли достоинства первых двух карт в перемешанной колоде.
        \item При совпадении увеличивает локальный счётчик успешных исходов.
        \item После завершения всех раундов обновляет глобальные счётчики\\ \texttt{total\_successes\_rounds} и \texttt{total\_rounds} под защитой мьютекса.
    \end{itemize}
\end{itemize}
