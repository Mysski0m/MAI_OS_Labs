\section{Выводы}
В ходе выполнения лабораторной работы были приобретены практические навыки в организации многопоточной обработки данных в операционных системах, синхронизации потоков и безопасной работы с разделяемыми ресурсами.\\
Была составлена и отлажена программа на языке С++, реализующая параллельное моделирование методом Монте-Карло для оценки вероятности совпадения достоинств двух верхних карт в случайно перетасованной колоде. Программа использует стандартные средства многопоточности C++ (\texttt{std::thread, std::mutex}), что обеспечивает корректную работу на операционных системах семейства Unix (включая Linux) и поддерживает кроссплатформенность.\\
В результате работы программа запускает указанное пользователем количество потоков, каждый из которых независимо выполняет часть общего числа экспериментов. Обмен данными между потоками сведён к минимуму: результаты агрегируются в глобальные счётчики под защитой мьютекса, что исключает гонки данных и гарантирует корректность итогового результата.\\
Были обработаны возможные ошибки ввода (некорректные аргументы командной строки), а также обеспечена корректная инициализация и завершение всех потоков. Экспериментально подтверждена эффективность параллельных вычислений: время выполнения сокращается почти пропорционально числу потоков до достижения аппаратного предела (числа логических ядер процессора).
\pagebreak