\section{Метод решения}

Алгоритм решения задачи:

\begin{enumerate}
\item Пользователь в консоль родительского процесса вводит имя файла, в который дочерний процесс запишет результат вычислений, а затем — последовательность чисел типа \texttt{float}, разделённых пробелами.

\item Создаётся объект логики родительского процесса (реализован в \texttt{parent.cpp}), инициализирующий работу основной программы.

\item Родительский процесс создаёт два канала (pipe).
    \begin{itemize}
    \item \texttt{input\_pipe} — для передачи последовательности чисел от родителя к дочернему процессу;
    \item \texttt{err\_pipe} — для передачи диагностических сообщений (ошибок) от дочернего процесса к родителю.
    Затем вызывается \texttt{fork()} для создания дочернего процесса.
    \end{itemize}
\item В контексте дочернего процесса:
    \begin{itemize}
    \item закрывается записывающий конец \texttt{input\_pipe} и читающий конец \texttt{err\_pipe};
    \item читающий конец \texttt{input\_pipe} перенаправляется на стандартный поток ввода (\texttt{stdin}) с помощью \texttt{dup2()};
    \item записывающий конец \texttt{err\_pipe} перенаправляется на стандартный поток ошибок (\texttt{stderr});
    \item выполняется системный вызов \texttt{execl()}, запускающий отдельный исполняемый файл \texttt{child}, которому передаётся имя выходного файла как аргумент командной строки.
    Если запуск дочернего исполняемого файла не удался, процесс завершается с ошибкой.
    \end{itemize}

\item Родительский процесс закрывает читающий конец \texttt{input\_pipe} и записывающий конец \texttt{err\_pipe}, после чего отправляет в \texttt{input\_pipe} строку с числами, введёнными пользователем.

\item Дочерний процесс (\texttt{child}) получает имя выходного файла через аргумент командной строки, считывает числа из \texttt{stdin} (который перенаправлен на \texttt{input\_pipe}), выполняет последовательное деление первого числа на все последующие. Перед каждым делением проверяется, не равен ли делитель нулю. В случае деления на ноль процесс немедленно завершается с выводом сообщения об ошибке в \texttt{stderr}.

\item Если деление выполнено успешно, дочерний процесс открывает (создаёт) указанный файл и записывает в него результат вычислений с точностью до шести знаков после запятой.

\item Родительский процесс ожидает завершения дочернего с помощью \texttt{wait()}. Если дочерний процесс завершился с ненулевым кодом (например, из-за деления на ноль), родитель считывает сообщение об ошибке из \texttt{err\_pipe} и также завершает свою работу. В случае успешного завершения программа завершается корректно.
\end{enumerate}

\vspace{10\baselineskip}

Архитектура программы:

\dirtree{%
.1 lab1-var4/. 
.2 bin/. 
.2 child.cpp.
.2 build/.
.2 include/.
.3 child.h.
.3 exceptions.h.
.3 os.h.
.3 parent.h.
.2 src/.
.3 child.cpp.
.3 os.cpp.
.3 parent.cpp.
.2 main.cpp.
}

Ссылки:

\begin{itemize}
\item \url{https://pubs.opengroup.org/onlinepubs/009696799/functions/write.html}
\item \url{https://pubs.opengroup.org/onlinepubs/009696799/functions/execl.html}
\item \url{https://pubs.opengroup.org/onlinepubs/009696799/functions/getppid.html}
\item \url{https://pubs.opengroup.org/onlinepubs/009696799/functions/getpid.html}
\item \url{https://pubs.opengroup.org/onlinepubs/009696799/functions/waitpid.html}
\item \url{https://pubs.opengroup.org/onlinepubs/009696799/functions/sleep.html}
\item \url{https://pubs.opengroup.org/onlinepubs/009696799/functions/exit.html}
\item \url{https://pubs.opengroup.org/onlinepubs/009696799/functions/kill.html}
\item \url{https://pubs.opengroup.org/onlinepubs/009696799/functions/dup2.html}
\item \url{https://pubs.opengroup.org/onlinepubs/009696799/functions/pipe.html}
\item \url{https://pubs.opengroup.org/onlinepubs/009696799/functions/fork.html}
\item \url{https://pubs.opengroup.org/onlinepubs/009696799/functions/close.html}
\end{itemize}

\section{Описание программы}

\texttt{main.cpp} --- точка входа в родительский процесс. Вызывает функцию \texttt{RunParentProcess()}, реализующую основную логику взаимодействия с пользователем и дочерним процессом.

\texttt{bin/child.cpp} --- точка входа в дочерний процесс. Принимает имя выходного файла как аргумент командной строки и вызывает функцию \texttt{RunChildProcess()} для выполнения вычислений.

\texttt{exceptions.h} --- объявление пользовательских исключений.

\texttt{os.h} --- объявление функций-обёрток над системными вызовами операционной системы (для поддержки кроссплатформенности). \\
\texttt{src/os.cpp} --- реализация функций. \\

Основные функции:
\begin{itemize} 
    \item \texttt{int CreatePipe(pipe\_t fd[2]);} --- создаёт анонимный канал. Используется системный вызов \texttt{pipe()}.
    \item \texttt{pid\_t CloneProcess();} --- создаёт копию текущего процесса. Используется системный вызов \texttt{fork()}.
    \item \texttt{int LinkStdinWithPipe(pipe\_t pipe);} --- перенаправляет стандартный ввод на указанный канал. Используется системный вызов \texttt{dup2()}.
    \item \texttt{int LinkStderrWithPipe(pipe\_t pipe);} --- перенаправляет стандартный поток ошибок на указанный канал. Используется системный вызов \texttt{dup2()}.
    \item \texttt{int ClosePipe(pipe\_t pipe);} --- закрывает файловый дескриптор канала. Используется системный вызов \texttt{close()}.
    \item \texttt{int Exec(const char* path);} --- заменяет текущий процесс на новый исполняемый файл. Используется системный вызов \texttt{execl()}.
    \item \texttt{int WaitForChild();} --- ожидает завершения любого дочернего процесса. Используется системный вызов \texttt{wait()}.
    \item \texttt{int WritePipe(pipe\_t pipe, void* buffer, size\_t bytes);} --- записывает данные в канал. Используется системный вызов \texttt{write()}.
    \item \texttt{int ReadPipe(pipe\_t pipe, void* buffer, size\_t bytes);} --- читает данные из канала. Используется системный вызов \texttt{read()}.
    \item \texttt{void AddSignalHandler(signal\_t sig, SignalHandler\_t handler);} --- регистрирует обработчик сигнала. Используется системный вызов \texttt{signal()}.
    \item \texttt{int ReadFromStdin(char* buffer, size\_t size);} --- читает данные напрямую из стандартного ввода. Используется системный вызов \texttt{read()}.
    \item \texttt{void PrintLastError();} --- выводит описание последней ошибки ОС. Используется функция \texttt{perror()}.
\end{itemize}

\vspace{1\baselineskip}

\texttt{child.h} --- объявление функции логики дочернего процесса. \\
\texttt{src/child.cpp} --- реализация вычислительной логики. \\
Основные функции:
\begin{itemize}
    \item \texttt{void RunChildProcess(const char* output\_file);} --- считывает числа из \texttt{stdin} (перенаправленного на канал), выполняет последовательное деление первого числа на все последующие, проверяет деление на ноль. При успехе записывает результат в файл \texttt{output\_file} с точностью до шести знаков после запятой. При ошибке завершает процесс с кодом 1.
\end{itemize}

\vspace{2\baselineskip}
\texttt{parent.h} --- объявление функции логики родительского процесса. \\
\texttt{src/parent.cpp} --- реализация управления процессами и взаимодействия с пользователем. \\
Основные функции:
\begin{itemize}
    \item \texttt{void RunParentProcess();} --- запрашивает у пользователя имя выходного файла и последовательность чисел, создаёт два канала (\texttt{input\_pipe} и \texttt{err\_pipe}), порождает дочерний процесс через \texttt{fork()}, перенаправляет его \texttt{stdin} и \texttt{stderr} на соответствующие концы каналов, запускает исполняемый файл \texttt{child} с передачей имени файла как аргумента, отправляет числа в \texttt{input\_pipe}, ожидает завершения дочернего процесса и обрабатывает возможные ошибки.
    \item \texttt{void OnChildKilled(signal\_t signum);} --- обработчик сигнала \texttt{SIGCHLD}. При завершении дочернего процесса читает диагностическое сообщение из \texttt{err\_pipe} и завершает родительский процесс.
    \item \texttt{void PrintErrorFromChild(pipe\_t pipe);} --- читает и выводит ошибки, полученные от дочернего процесса через канал.
\end{itemize}