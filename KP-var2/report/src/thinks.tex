\section{Выводы}

В ходе выполнения лабораторной работы были приобретены и закреплены практические навыки системного программирования в операционной системе Linux, а именно проектирование и реализация межпроцессного взаимодействия (IPC).

Были решены следующие задачи:
\begin{itemize}
    \item Освоение \texttt{Named Pipes}: Изучен механизм работы с именованными каналами (\texttt{mkfifo}, \texttt{open}, \texttt{read}, \texttt{write}). Реализована надежная схема обмена данными, предотвращающая блокировки (\texttt{deadlocks}) за счет правильного выбора режимов открытия файлов (\texttt{O\_RDWR} на сервере) и архитектуры «один общий канал на вход — уникальные каналы на выход».
    \item Сетевая архитектура: Реализована топология «Звезда», где сервер выступает в роли координатора и арбитра. Это позволило централизовать логику игры, упростить клиенты до уровня терминалов ввода-вывода и исключить возможность жульничества (так как карта противника хранится только на сервере).
    \item Многопоточное программирование: Применение \texttt{POSIX Threads} позволило реализовать асинхронный ввод-вывод на клиенте, разделив задачи ожидания сети и взаимодействия с пользователем, что является стандартом для современных сетевых приложений.
    \item Декомпозиция и модульность: Проект разделен на логические модули (сетевая обертка, протокол, игровая логика), что упрощает его поддержку и масштабирование.
    \item Разработанный программный комплекс полностью удовлетворяет требованиям задания, демонстрирует стабильную работу при сценарии игры двух пользователей и подтверждает эффективность выбранных методов \texttt{IPC} для решения задач локального клиент-серверного взаимодействия.
\end{itemize}
\pagebreak