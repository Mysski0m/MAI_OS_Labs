\section{Условие}

{\bfseries Цель курсового проекта:}
\begin{itemize}
\item Приобретение практических навыков в использовании знаний, полученных в течении курса
\item Проведение исследования в выбранной предметной области
\end{itemize}

{\bfseries Задание:} \\
Необходимо спроектировать и реализовать программный прототип в соответствии с выбранным вариантом. Произвести анализ и сделать вывод на основании данных, полученных при работе программного прототипа. \\
Проектирование консольной клиент-серверной игры\\
На основе любой из выбранных технологий:
\begin{itemize}
    \item Pipes
    \item Sockets
    \item Сервера очередей
    \item И другие
\end{itemize}
Создать собственную игру более, чем для одного пользователя. Игра может быть устроена по принципу: клиент-клиент, сервер-клиент. \\

{\bfseries Вариант:} 2 \\
Консоль-серверная игра. Необходимо написать консоль-серверную игру. Необходимо написать 2 программы: сервер и клиент. Сначала запускается сервер, а далее клиенты соединяются с сервером. Сервер координирует клиентов между собой. При запуске клиента игрок может выбрать одно из следующих действий (возможно больше, если предусмотрено вариантом):
\begin{itemize}
    \item Создать игру, введя ее имя
    \item Присоединиться к одной из существующих игр по имени игры
\end{itemize}
Морской бой. Общение между сервером и клиентом необходимо организовать при помощи \texttt{pipe'ов}. Каждый игрок должен при запуске ввести свой логин. Должна быть предоставлена возможность отправить приглашение на игру другому игроку по логину
\pagebreak