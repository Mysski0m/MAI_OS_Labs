\section{Результаты}

В ходе работы спроектирован и реализован программный прототип многопользовательской игры «Морской бой», использующий архитектуру «клиент-сервер».

Взаимодействие процессов: Успешно организована передача структурированных данных (пакетов \texttt{struct Packet}) между независимыми процессами сервера и нескольких клиентов через именованные каналы (\texttt{Named Pipes/FIFO}). Сервер корректно обрабатывает запросы от нескольких клиентов, используя мультиплексирование сообщений из единого канала, а клиенты асинхронно получают ответы через персональные каналы.

Игровая логика: Реализована полноценная логика игры: регистрация пользователей, механизм приглашений (\texttt{/invite}), автоматическая генерация игровых полей, поочередная стрельба, проверка попаданий и определение победителя. Сервер корректно управляет состоянием игры, блокируя попытки хода вне очереди или выстрелы по уже пораженным клеткам.

Синхронизация и многопоточность: В клиентском приложении успешно применена многопоточность с использованием библиотеки \texttt{pthread}. Фоновый поток обеспечивает непрерывное чтение сообщений от сервера (чат, обновление поля), не блокируя основной поток ввода команд пользователем. На сервере обеспечена потокобезопасность списка игроков с помощью мьютексов (\texttt{pthread\_mutex}).

Пользовательский интерфейс: Реализована система визуализации игрового поля в консоли. Игроки получают актуальное отображение своего флота (с повреждениями) и «радар» поля противника (скрывающий корабли, но отображающий результаты выстрелов), что обеспечивает комфортный игровой процесс.

Обработка исключительных ситуаций: Программа корректно обрабатывает выход игроков (команда \texttt{/quit}), автоматически присуждая победу оставшемуся сопернику и удаляя отключившегося пользователя из списков сервера. Ресурсы (файлы каналов в \texttt{/tmp}) корректно освобождаются при завершении работы приложений.
